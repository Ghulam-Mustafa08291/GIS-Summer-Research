\documentclass[preprint,12pt]{elsarticle}

\usepackage{amssymb}
\usepackage{amsmath}
\usepackage{float} % Add this package for better figure control
\usepackage{xurl} % For better URL line breaking (allows breaks at any character)
\usepackage[breaklinks=true]{hyperref} % Makes URLs clickable even when wrapped


\journal{Remote Sensing Applications: Society and Environment}

\begin{document}

\begin{frontmatter}

\title{Early Warning System for District-Level Weather Anomalies in Pakistan Using Global Weather and Climate Datasets}

\author{Ghulam Mustafa}

\affiliation{organization={Habib University}, 
            city={Karachi},
            country={Pakistan}}

%% Abstract
\begin{abstract}
Pakistan currently has sparse coverage by high-quality weather observation stations, limiting the ability to develop localized early warning systems for weather-related disasters. This paper presents an approach for the development of an early warning system for district-level weather anomalies in Pakistan using only globally available weather, climate, and reanalysis datasets. The presented approach combines global data for historical weather patterns from the ERA5-Land Monthly Aggregated dataset with short-term weather forecasts from the Global Forecast System (GFS). The historical data is used to calculate climatological baselines (10-year averages from 2014--2024) for rainfall and temperature for each of Pakistan's 150+ districts. For early warning, observed weather data from the preceding 90 days (three months) is combined with the 16-day GFS forecast, and the combined behavior over this 90+16 day period is compared against the climatological baseline for the corresponding period. This integration of recent past weather data with short-term forecasts enables enhanced early warning capability, allowing users to identify not only current anomalous conditions but also anticipate emerging risks before they fully develop. The early warning system is implemented as an interactive Google Earth Engine application that presents results through a single map with three toggleable layers: (1) anomalies based on the past 90 days, (2) anomalies based on the 16-day forecast, and (3) combined anomalies integrating both periods. The paper uses the case study of the 2010 and 2022 Pakistan floods to explore both the potential and the limitations of this approach toward building an operational early warning system. Results demonstrate that the combined past+forecast view provides enhanced insight for identifying developing hazardous conditions compared to analyzing either period in isolation.
\end{abstract}

%% Graphical abstract
\begin{graphicalabstract}
% \includegraphics{graphical-abstract.png}
\end{graphicalabstract}

%% Research highlights
\begin{highlights}
\item Early warning system using global datasets (ERA5, GFS) to overcome Pakistan's sparse local weather station coverage.
\item Climatological baselines computed from 10-year ERA5-Land data (2014--2024) for each of 150+ districts.
\item Single GEE application with three map layers: past 90-day anomalies, 16-day forecast anomalies, and combined 90+16 day anomalies.
\item Combined past+forecast analysis provides enhanced early warning insight over isolated analysis.
\item Case study: 2010 and 2022 Pakistan floods demonstrate potential and limitations of the approach.
\end{highlights}

%% Keywords
\begin{keyword}
early warning system \sep climate anomalies \sep ERA5-Land \sep GFS forecast \sep Pakistan \sep Google Earth Engine \sep flood risk
\end{keyword}

\end{frontmatter}

%% ---------------- Main Text ----------------

\section{Introduction}

Pakistan is among the countries most vulnerable to climate change impacts. According to the Germanwatch Global Climate Risk Index, Pakistan has consistently ranked among the top 10 most affected countries, placed 5th in 2020, 8th in 2021, and identified as the most vulnerable country in 2022 following the catastrophic monsoon floods \cite{germanwatch2021}. The country has experienced increasingly frequent and severe extreme weather events, including the devastating 2010 floods and the unprecedented 2022 floods that affected over 33 million people and caused billions of dollars in damages to infrastructure, agriculture, and housing. Recent studies have documented escalating drought intensity across Pakistan's agro-ecological zones, with precipitation deficits of up to 63.5\% during critical crop development stages and significant warming trends \cite{iqbal2025drought}.

These climate vulnerabilities underscore the critical need for effective early warning systems that can help decision-makers anticipate emerging risks. Early detection of rainfall and temperature deviations from historical norms enables proactive responses: farmers can adjust planting schedules and irrigation practices, water managers can prepare for droughts or floods, and disaster agencies can issue timely warnings. However, Pakistan's weather observation infrastructure presents a significant challenge. The country has sparse coverage by high-quality weather observation stations, particularly in rural and mountainous areas where climate impacts are often most severe. This data scarcity limits the ability to develop localized early warning systems using conventional ground-based monitoring approaches.

Global climate datasets offer a promising alternative to overcome this limitation. The ERA5-Land reanalysis dataset from the European Centre for Medium-Range Weather Forecasts (ECMWF) provides consistent, gap-free historical weather data at approximately 9~km resolution, while the Global Forecast System (GFS) from the National Oceanic and Atmospheric Administration (NOAA) provides 16-day weather forecasts at 0.25° resolution. Both datasets are freely available through Google Earth Engine (GEE), a cloud-based platform for planetary-scale geospatial analysis that requires no local data storage or processing capabilities. By leveraging these global datasets and GEE's computational infrastructure, it becomes possible to develop district-level early warning tools that function independently of local observation networks, making climate intelligence accessible to a broader range of stakeholders across Pakistan.

This paper presents an early warning system for district-level weather anomalies in Pakistan that combines historical weather patterns from ERA5-Land with short-term forecasts from GFS. The system uses a 10-year baseline period (2014--2024) to calculate climatological averages for rainfall and temperature for each of Pakistan's 150+ districts. For early warning, the system compares observed weather from the preceding 90 days (three months) and the upcoming 16-day forecast against these baseline values. The early warning system is implemented as an interactive GEE application presenting results through a single map with three toggleable layers: (1) anomalies based on the past 90 days only, (2) anomalies based on the 16-day forecast only, and (3) combined anomalies integrating both periods. This combined view enables users to identify not only current anomalous conditions but also anticipate emerging risks before they fully develop. A case study of the 2010 and 2022 Pakistan floods demonstrates both the potential and the limitations of this approach toward building an operational early warning system.

\section{Literature Survey}
Several platforms and tools have been developed for climate monitoring and anomaly detection at various spatial scales. Recent research has explored data-driven approaches for climate risk assessment in Pakistan, integrating remote sensing with spatial analysis techniques \cite{fcsi2024climate}. This section reviews existing global platforms and Pakistan-specific tools, identifying gaps that motivate the present work.

\subsection{Global Platforms}
\textbf{Climate Engine} (\url{https://climateengine.org}) is a web-based platform built on Google Earth Engine that provides access to climate and remote sensing data including ERA5 (1979--present, 0.25° resolution), Landsat, and MODIS datasets. Developed through a collaboration between the Desert Research Institute, University of California Merced, and Google, the platform enables users to generate maps of average temperature and precipitation values over custom date ranges, compute deviations from long-term baseline means, and calculate drought indices such as SPI, SPEI, EDDI, and PDSI directly within the application. While the platform can display temperature and rainfall anomalies for Pakistan at the country level, it does not provide district-level analysis, district-specific time-series visualizations, or forecast-based anomaly detection using GFS data.

\textbf{FAO GIEWS Earth Observation} (\url{https://www.fao.org/giews/earthobservation/}) provides near real-time monitoring of agricultural conditions with updates every 10 days (dekadal). The platform offers precipitation anomaly maps showing differences between current rainfall and long-term averages, using data from the European Centre for Medium-Range Weather Forecasts (ECMWF). It also includes vegetation indices (NDVI, VCI, VHI), Agricultural Stress Index, and historic drought frequency analysis. While it covers Pakistan at the country level and provides precipitation anomalies, it does not offer temperature anomaly products or district-level spatial granularity.

\textbf{Climate Toolbox} (\url{https://climatetoolbox.org}) offers a suite of interactive web tools for mapping and analyzing historical and projected climate data across the United States. It uses GridMET and PRISM datasets to provide county-level analysis with forecasts extending up to 7 months, and baseline periods varying by tool: historical tools use 1981--2010 as the reference period, while future projection comparisons use 1971--2000. Similarly, the \textbf{West Wide Drought Tracker} (\url{https://wrcc.dri.edu/wwdt/}) provides monthly drought indices including Palmer Drought Severity Index (PDSI) as well as mean precipitation and temperature displayed as time-series graphs, using PRISM data for the western United States at county-level resolution. Both tools demonstrate the utility of sub-national climate monitoring with time-series capabilities but are geographically limited to the US.

\subsection{Pakistan-Specific Tools}
\textbf{ICIMOD National Agricultural Drought Watch for Pakistan} (\url{https://tethys.icimod.org/apps/droughtpk/}) was developed under the SERVIR Hindu Kush Himalaya Initiative and provides district-level agricultural drought monitoring. The platform uses CHIRPS precipitation data and MODIS imagery to compute indices including rainfall, mean temperature, evapotranspiration, soil moisture, and Standardized Precipitation Index (SPI). It offers three analysis modes: Current conditions, Seasonal analysis, and a 9-month Outlook forecast. The platform provides district-level time-series graphs displaying up to four indices simultaneously, with long-term average comparisons. The Outlook tab presents forecasted monthly anomalies as Z-scores (standardized deviations from the mean) displayed in box plots. While this tool offers comprehensive agricultural drought analysis, it differs from the present study in two key aspects: (1) it focuses on seasonal forecasts (9 months) rather than short-term operational forecasts (16 days), and (2) it does not provide a cumulative score combining past observations with forecast anomalies.

The \textbf{Pakistan Meteorological Department (PMD)} (\url{https://www.pmd.gov.pk}) provides official weather forecasts and climate data for Pakistan. Short-term forecasts (3-day) are provided at the provincial level, while monthly forecasts include district-level maps for both rainfall and temperature. The monthly products show baseline normal values (using 1991--2020 as the reference period) and anomaly maps indicating deviations from this baseline for each district. However, PMD does not provide interactive web-based applications for custom analysis, historical pattern exploration, or integration with GFS forecast data for 16-day anomaly detection.

The \textbf{NDMA Disaster Alert} mobile application developed by Pakistan's National Disaster Management Authority provides district-level alerts for various hazards including floods, heatwaves, earthquakes, and smog. While it offers district-level alert information and GIS-based mapping, it focuses exclusively on current disaster alerting rather than systematic anomaly analysis. The application does not provide historical time-series analysis, pattern visualization, forecast comparison with historical baselines, or cumulative anomaly scoring.

The review of existing tools reveals that while several platforms provide valuable climate monitoring capabilities, none offers a comprehensive early warning system for Pakistan that addresses three critical needs simultaneously. First, no existing tool relies exclusively on globally available datasets, making it difficult to develop a replicable and scalable system independent of sparse local observation networks. The approach presented in this study depends entirely on global reanalysis (ERA5-Land) and forecast (GFS) datasets, ensuring that the system can function without local ground-station data and can potentially be adapted to other data-sparse regions. Although GFS 16-day rainfall forecasts have been validated for agricultural advisory in Pakistan \cite{bhatti2022gfs}, no interactive tool currently leverages these forecasts for district-level early warning.

Second, existing tools analyze historical conditions or forecast data in isolation. No tool combines recent past weather observations with short-term forecasts to provide a unified early warning view. The system developed in this study integrates observed weather data from the preceding 90 days with the 16-day GFS forecast, enabling users to identify not only current anomalous conditions but also anticipate emerging risks---a capability unavailable in any existing Pakistan-focused platform.

Third, existing tools offer limited customizability and interactivity. The system presented here allows users to interactively explore anomalies at the district level and provides toggleable map layers for past, forecast, and combined views. The performance and usefulness of this approach is demonstrated through comparison with independent flooding data for the 2010 and 2022 Pakistan floods.

\section{Data and Methods}
\subsection{Datasets}
This study utilizes two primary datasets for climate analysis and forecasting:

\textbf{ERA5-Land Monthly Aggregated Dataset (1981–present):} ERA5-Land is a reanalysis dataset produced by the European Centre for Medium-Range Weather Forecasts (ECMWF). It provides global land surface variables at approximately 9~km (0.1°) spatial resolution. ERA5 has been validated against surface gauge observations over Pakistan, demonstrating strong correlations (0.87--0.98) across monthly, seasonal, and annual timescales, supporting its use for climate monitoring applications in the region \cite{ahmed2022era5}. This study uses the monthly aggregated version available through Google Earth Engine, specifically extracting the \texttt{total\_precipitation\_sum} (total monthly precipitation in meters) and \texttt{temperature\_2m} (air temperature at 2 meters above the surface in Kelvin) bands. The dataset's long temporal coverage (1981–present) enables robust climatological baseline calculations.

\textbf{GFS: Global Forecast System 384-Hour Predicted Atmosphere Data:} The GFS is a global numerical weather prediction system operated by the National Oceanic and Atmospheric Administration (NOAA). It provides forecast data at 0.25° spatial resolution with predictions extending up to 384 hours (16 days) into the future. GFS rainfall forecasts have been statistically verified for agricultural applications in Pakistan, demonstrating useful prediction skill for farmers' decision-making \cite{bhatti2022gfs}. This study uses the \texttt{precipitation\_rate} (kg/m²/s) and \texttt{temperature\_2m\_above\_ground} (K) variables. Hourly forecasts are available for the first 120 hours, followed by 3-hourly forecasts for hours 123--384.

\subsection{Study Area}
The study area encompasses all districts of Pakistan, providing district-level spatial resolution for climate anomaly analysis. Pakistan's diverse geography includes multiple climate zones ranging from arid and semi-arid regions in the south and west to humid subtropical and alpine climates in the north. This climatic diversity makes district-level analysis essential for capturing localized weather patterns and anomalies.

\subsection{Climatological Baseline Calculation}
The early warning system requires climatological baselines against which current and forecast conditions can be compared. For each of Pakistan's 150+ districts, monthly average values for rainfall and temperature are computed from the ERA5-Land Monthly Aggregated dataset using a 10-year reference period (2014--2024). This baseline represents the expected climatological values for each month and each district. The baselines are pre-computed and stored as a Google Earth Engine Feature Collection asset to enable rapid real-time comparison without requiring repeated historical data processing.

\subsection{Anomaly Calculation Methodology}
The system computes anomalies as the direct difference between observed/forecast values and the corresponding baseline values:
\begin{equation}
Anomaly = Value_{observed/forecast} - Value_{baseline}
\end{equation}

This difference-based approach expresses anomalies in intuitive units (millimeters for rainfall, degrees Celsius for temperature) that are immediately understandable to farmers, water managers, and disaster management authorities without requiring statistical interpretation.

\subsection{Early Warning System Application}
The early warning system is implemented as an interactive Google Earth Engine application that presents results through a single map with three toggleable layers. Users can select a weather parameter (rainfall or temperature) and visualize the anomaly conditions across all districts of Pakistan.

\subsubsection{Layer 1: Past 90-Day Anomaly}
This layer displays the cumulative anomaly based on observed weather data from the preceding 90 days (three months) from the ERA5-Land dataset. The application:
\begin{enumerate}
    \item Retrieves the three most recent months of ERA5 data for each district.
    \item Computes the difference between each month's observed value and its corresponding baseline value.
    \item Sums these three monthly deviations to produce a cumulative 90-day anomaly score for each district.
\end{enumerate}

\subsubsection{Layer 2: 16-Day Forecast Anomaly}
This layer displays the anomaly based on the upcoming 16-day weather forecast from the GFS dataset. The application:
\begin{enumerate}
    \item Retrieves the latest GFS forecast data covering the next 16 days (384 hours).
    \item For precipitation, aggregates hourly (hours 1--120) and three-hourly (hours 123--384) precipitation rates across all forecast time steps. For temperature, computes the mean temperature across all forecast time steps.
    \item Handles month overlapping: since the 16-day forecast period may span two calendar months, the application calculates a weighted historical baseline. It determines the number of days falling within each month, retrieves the corresponding monthly baseline values, and computes a proportionally weighted average to represent the expected 16-day baseline.
    \item Computes the difference between the 16-day forecast value and the 16-day equivalent baseline for each district.
\end{enumerate}

\subsubsection{Layer 3: Combined 90+16 Day Anomaly}
This layer displays the combined anomaly by integrating the past 90-day observations with the 16-day forecast. The application:
\begin{enumerate}
    \item Computes the 90-day past anomaly as described in Layer 1.
    \item Computes the 16-day forecast anomaly as described in Layer 2.
    \item Sums these two values to produce a cumulative anomaly score representing the combined deviation over the 90+16 day period.
\end{enumerate}

This combined view is the key innovation of the early warning system, as it enables users to identify not only current anomalous conditions but also anticipate whether those conditions are expected to persist, intensify, or diminish in the coming 16 days.

\subsection{Visualization and Interaction}
Districts are color-coded on the map based on their anomaly values using a continuous color scale:
\begin{itemize}
    \item \textbf{Blue}: Values below the historical baseline (deficit conditions for rainfall, cooler for temperature).
    \item \textbf{Green}: Values near the historical baseline (normal conditions).
    \item \textbf{Red}: Values above the historical baseline (excess conditions for rainfall, warmer for temperature).
\end{itemize}

The color scale is dynamically adjusted based on the maximum absolute anomaly value across all districts to ensure optimal visual contrast. Users can click on any district to view detailed information including the district name, observed/forecast value, baseline value, and the computed anomaly.

\subsection{Technical Implementation}
The application is implemented using the Google Earth Engine JavaScript API. To handle the computational load of processing 150+ districts, the application employs batch processing, computing anomalies for districts in batches of 5 with a short delay between batches to prevent browser timeouts. The application dynamically retrieves the latest available ERA5 and GFS data at runtime, ensuring that users always see the most current information available.

\section{Results}

The early warning system application provides three complementary views of weather anomalies across Pakistan's districts. Figure~\ref{fig:three-layers} presents all three layers side-by-side, demonstrating how the combined view provides enhanced insight compared to analyzing either past conditions or forecasts in isolation.

\subsection{Layer 1: Past 90-Day Anomaly}
The past 90-day anomaly layer shows the cumulative deviation of observed weather conditions over the preceding three months from the climatological baseline. Districts shown in red indicate above-average conditions (excess rainfall or higher temperatures), while districts shown in blue indicate below-average conditions (deficit rainfall or lower temperatures). Green districts are experiencing near-normal conditions.

\subsection{Layer 2: 16-Day Forecast Anomaly}
The forecast anomaly layer shows how the upcoming 16-day weather forecast compares to the climatological baseline for that period. This layer provides the forward-looking early warning capability of the system, allowing users to anticipate districts where conditions are expected to deviate from normal in the coming days.

\subsection{Layer 3: Combined 90+16 Day Anomaly}
The combined anomaly layer integrates the past 90 days of observed conditions with the upcoming 16-day forecast. This cumulative view is the key innovation of the early warning system. By showing the total deviation over the 90+16 day period, this layer enables users to identify:
\begin{itemize}
    \item Districts with persistent anomalous conditions that are expected to continue
    \item Districts where current conditions may intensify or moderate based on the forecast
    \item The overall trajectory of weather conditions relative to historical norms
\end{itemize}

\begin{figure}[H]
    \centering
    % TODO: Replace with new 3-layer screenshot from single EWS app
    \includegraphics[width=\textwidth]{combined_forecast_historical.png}
    \caption{Three-layer visualization from the Early Warning System application showing (left) past 90-day rainfall anomaly, (center) 16-day forecast anomaly, and (right) combined 90+16 day anomaly for Pakistan districts. Blue indicates below-average conditions, green indicates near-average conditions, and red indicates above-average conditions.}
    \label{fig:three-layers}
\end{figure}

\subsection{Enhanced Insight from Combined Analysis}
Comparing the three layers reveals districts where the combined view provides additional insight beyond the individual layers:
\begin{enumerate}
    \item \textbf{Persistent High-Risk Districts}: Districts showing red in both the past and forecast layers appear intensely red in the combined layer, highlighting areas of sustained above-average conditions requiring attention.
    \item \textbf{Transitioning Conditions}: Districts showing different colors between past and forecast layers (e.g., red past but blue forecast) may appear green in the combined layer, indicating conditions are expected to normalize.
    \item \textbf{Emerging Risks}: Districts showing green in the past but red in the forecast indicate newly developing anomalous conditions that warrant early warning.
\end{enumerate}

\section{Case Study: Pakistan Floods (2010 and 2022)}

To evaluate the potential and limitations of the early warning system approach, we examine two of Pakistan's most devastating flood events: the 2010 and 2022 monsoon floods.

\subsection{Background}
The 2010 Pakistan floods were among the most destructive in the country's history, affecting over 20 million people and causing widespread damage across Punjab and Sindh provinces. The 2022 floods were even more catastrophic, affecting over 33 million people and causing an estimated \$30 billion in damages. Both events were characterized by anomalously high rainfall during the monsoon season.

\subsection{Retroactive Analysis}
Using the early warning system methodology, we analyze the rainfall anomaly patterns in the months preceding and during these flood events. For each event, we examine:
\begin{itemize}
    \item The past 90-day rainfall anomaly leading up to the flood period
    \item The 16-day forecast anomaly (using GFS data availability for 2022)
    \item The combined anomaly pattern that would have been visible to users
\end{itemize}

\textit{[Detailed analysis with figures to be added based on new single app implementation]}

\subsection{Potential and Limitations}
The case study reveals both the potential and limitations of this early warning approach:

\textbf{Potential:}
\begin{itemize}
    \item District-level anomaly detection successfully identifies areas receiving significantly above-normal rainfall
    \item The combined 90+16 day view captures persistent excess rainfall conditions
    \item The approach works independently of ground-based observation networks
\end{itemize}

\textbf{Limitations:}
\begin{itemize}
    \item Monthly temporal resolution may miss short-term extreme events within a month
    \item 16-day forecasts have decreasing skill at longer lead times
    \item The system identifies anomalies but does not directly predict flood extent or impact
\end{itemize}

\subsection{Validation Against Independent Flood Data}
[PLACEHOLDER: This section will present a quantitative comparison of the system's anomaly detections with independent flood extent/standing water datasets for the 2010 and 2022 Pakistan floods. The comparison will assess the system's ability to correctly identify districts that experienced flooding, providing an independent measure of the early warning system's performance.]

\section{Discussion}

The early warning system developed in this study demonstrates the feasibility of using globally available climate datasets to provide district-level weather anomaly analysis for Pakistan. By combining historical ERA5 data with GFS forecasts, the system addresses a critical gap in Pakistan's climate monitoring infrastructure.

\subsection{Key Findings}
The three-layer visualization approach reveals that the combined past+forecast view provides enhanced insight compared to analyzing either period in isolation. Districts experiencing persistent anomalous conditions are clearly identified in the combined layer, enabling more informed decision-making for disaster preparedness, agriculture planning, and water resource management.

\subsection{Comparison with Existing Tools}
Unlike existing platforms such as ICIMOD DroughtPK that focus on seasonal (9-month) outlooks, this system provides short-term operational forecasts (16 days) that are actionable for immediate early warning. The use of intuitive units (mm, °C) rather than standardized Z-scores makes the results accessible to non-technical stakeholders including farmers and local disaster management authorities.

\subsection{Limitations}
Several limitations should be noted:
\begin{enumerate}
    \item \textbf{Temporal Resolution}: The use of monthly ERA5 data limits the ability to capture intra-month variability. Daily or hourly data could provide finer temporal resolution but would require additional computational resources.
    \item \textbf{Baseline Period}: The 10-year baseline (2014--2024) is shorter than the WMO-recommended 30-year climatological normal. This was chosen based on data availability in the pre-computed asset, but a longer baseline would provide more robust climatological averages.
    \item \textbf{Single Parameter Analysis}: The current system analyzes rainfall and temperature independently. Future versions could incorporate multi-parameter indices that combine multiple climate variables.
\end{enumerate}

\subsection{Future Work}
To move toward an operational early warning system, several improvements are recommended:
\begin{itemize}
    \item Development of threshold-based alert levels (e.g., yellow/orange/red warnings)
    \item Extension to finer temporal resolution using daily ERA5 data
    \item Integration with impact models to translate weather anomalies into estimated flood risk
\end{itemize}

\section{Conclusion}

This paper presents an early warning system for district-level weather anomalies in Pakistan that addresses the challenge of sparse local weather observation networks by leveraging globally available climate datasets. The system combines historical weather patterns from the ERA5-Land dataset with 16-day forecasts from the Global Forecast System to provide a three-layer view of weather anomalies: past 90-day conditions, 16-day forecast, and combined 90+16 day cumulative anomalies.

The key contribution of this work is the demonstration that combining recent past observations with short-term forecasts provides enhanced early warning capability compared to analyzing either period in isolation. The combined view enables users to identify districts with persistent anomalous conditions, anticipate emerging risks, and track the trajectory of weather conditions relative to climatological norms.

The system is implemented as an interactive Google Earth Engine application that is freely accessible and requires no local data storage or processing capabilities. This accessibility is particularly important for developing regions where computational resources may be limited but climate risks are high.

The case study of the 2010 and 2022 Pakistan floods demonstrates the system's ability to detect anomalous rainfall patterns at the district level, and comparison with independent flood data provides an initial assessment of the system's early warning performance. Future work should focus on integration with impact models and development of operational alert thresholds to support disaster management authorities in issuing timely warnings.

%% ---------------- References ----------------
\begin{thebibliography}{00}

\bibitem{germanwatch2021}
  Eckstein, D., K\"unzel, V., \& Sch\"afer, L.,
  \textit{Global Climate Risk Index 2021},
  Germanwatch e.V., Bonn,
  2021.
  Available: https://www.germanwatch.org/en/cri

\bibitem{iqbal2025drought}
  Iqbal, M.F., et al.,
  \textit{Evolving climatic patterns and drought risk in Pakistan's agro-ecological zones},
  Theoretical and Applied Climatology,
  2025.
  DOI: 10.1007/s00704-025-05681-y

\bibitem{fcsi2024climate}
  Authors,
  \textit{Integrating Remote Sensing and Spatial Artificial Intelligence for Climate Risk Assessment in Pakistan: A Data-Driven Spatiotemporal Analysis},
  Frontiers in Computational and Smart Intelligence (FCSI),
  2024.

\bibitem{ahmed2022era5}
  Ahmed, K., Sachindra, D.A., Shahid, S., Iqbal, Z., Nawaz, N., \& Khan, N.,
  \textit{Performance evaluation and comparison of observed and reanalysis gridded precipitation datasets over Pakistan},
  Theoretical and Applied Climatology,
  vol. 150, pp. 1--21,
  2022.
  DOI: 10.1007/s00704-022-04100-w

\bibitem{bhatti2022gfs}
  Bhatti, M.T. \& Anwar, A.A.,
  \textit{Statistical verification of 16-day rainfall forecast for a farmers advisory service in Pakistan},
  Agricultural and Forest Meteorology,
  vol. 318, 108888,
  2022.
  DOI: 10.1016/j.agrformet.2022.108888

\end{thebibliography}

\end{document}
